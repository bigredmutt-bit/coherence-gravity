\documentclass[11pt]{article}

\usepackage{amsmath,amssymb,amsfonts}
\usepackage{geometry}
\usepackage{hyperref}
\usepackage{bm}
\usepackage{setspace}
\geometry{margin=1in}
\setstretch{1.15}

\title{Observational Predictions of Coherence--Field Gravity:\\
Galaxies, Clusters, and the Ultra-Weak Regime}

\author{
Clifford Treadwell\\[0.5em]
{\normalsize with model-assisted analysis generated using the GPT-5.1 system}
}
\date{\today}

\begin{document}
\maketitle

\begin{abstract}
This technical note summarizes the observable predictions of
Coherence--Field Gravity (CFG) across galactic and cluster scales.
CFG introduces a coherence field $C(r)$ whose gradient produces a universal
additional acceleration term $A/r$ in baryonic environments. Combined with
the Newtonian term, this yields
\[
a(r)=\frac{GM(r)}{r^{2}}+\frac{A}{r}.
\]
We outline the empirical signatures distinguishing CFG from particle dark
matter and MOND-like theories. Core predictions include a universal
transition radius $r_{t}\approx 0.30\,\mathrm{kpc}$, a fixed mass scale
$M_0\approx5.4\times10^{7}\,M_{\odot}$, and distinct behavior in clusters,
wide binaries, gravitational lensing, and the ultra-weak acceleration regime.
\end{abstract}

\section{Introduction}

Coherence--Field Gravity replaces dark matter with a scalar field whose
gradient contributes an $A/r$ acceleration term. This note summarizes the
observable consequences of that structure.

Predictions here are model-independent within CFG and derived without
fine-tuned galaxy-by-galaxy parameters.

\section{Galactic Rotation Curves}

Given baryonic mass $M(r)$, CFG predicts:
\[
a(r)=a_{N}(r)+\frac{A}{r},
\qquad
a_{N}(r)=\frac{GM(r)}{r^{2}}.
\]

Key signatures:
\begin{itemize}
    \item flat rotation curves emerge naturally from the $1/r$ term,
    \item inner regions remain Newtonian,
    \item logarithmic potential contribution $\Phi\propto\ln r$,
    \item reduced scatter in baryonic Tully--Fisher relation,
    \item no need for halo concentration or feedback tuning.
\end{itemize}

\subsection{Transition Radius}

CFG predicts a universal radius where
\[
a_{N}(r_t)\simeq \frac{A}{r_t}.
\]

SPARC data analysis indicates
\[
r_t\approx 0.30\,\mathrm{kpc}.
\]

This radius is independent of galaxy mass or morphology.

\section{Cluster-Scale Predictions}

CFG naturally yields higher accelerations in clusters due to their larger
baryonic mass and decoherence-weighted sourcing.

Predicted signatures:
\begin{itemize}
    \item velocity dispersions consistent with observed cluster masses,
    \item stronger apparent ``dark mass'' effect than galaxies,
    \item no need for massive cold dark matter halos,
    \item consistency with X-ray inferred mass profiles,
    \item deviations from NFW-like behavior in the inner region.
\end{itemize}

\section{Gravitational Lensing Predictions}

CFG predicts lensing through the modified potential
\[
\Phi(r)= -\frac{GM(r)}{r} + A\ln r.
\]

Consequences:
\begin{itemize}
    \item lensing arcs stronger than pure baryons predict,
    \item similar magnitudes to dark matter lensing in clusters,
    \item but shallower shear profiles at large radius than NFW,
    \item testable difference: logarithmic potential vs. $1/r$ halo.
\end{itemize}

\section{Wide Binary Stars}

In the ultra-weak regime:
\[
a(r)\ll 10^{-11}\,\mathrm{m/s^{2}},
\]
CFG predicts enhanced accelerations from the $1/r$ term.

Expected signatures:
\begin{itemize}
    \item velocity dispersions slightly larger than Newtonian,
    \item no sharp MOND-like cutoff,
    \item smooth transition at $r_t$, not at a specific acceleration scale.
\end{itemize}

These predictions can be tested using Gaia DR4+.

\section{Faint Dwarf Galaxies}

CFG predicts:
\begin{itemize}
    \item reduced scatter in dwarf galaxy velocity curves,
    \item no need for cored versus cuspy halo debates,
    \item baryon-driven structure even at low mass.
\end{itemize}

Ultra-faint dwarfs may show measurable deviations from $\Lambda$CDM halo
predictions.

\section{Ultra-Weak Regime Behavior}

CFG distinguishes itself from MOND by predicting:
\begin{itemize}
    \item $1/r$ acceleration persists without a hard acceleration scale,
    \item inner regions remain fully Newtonian,
    \item no violation of strong equivalence principle at laboratory scales,
    \item smooth interpolation from Newtonian to coherence-dominated regimes.
\end{itemize}

\section{Comparison to Dark Matter and MOND}

CFG differs from:
\begin{itemize}
    \item \textbf{Cold Dark Matter (CDM):} no halos, no substructure issues, no cusp-core tension.
    \item \textbf{MOND:} no $a_0$ threshold, no fixed acceleration scale, no deep-MOND limit.
    \item \textbf{Scalar-Tensor Models:} does not require galaxy-by-galaxy tuning.
\end{itemize}

\section{Conclusions}

The acceleration structure of CFG yields a suite of observational consequences
that naturally explain galactic and cluster dynamics while remaining
consistent with lensing and wide binary data. These predictions distinguish
CFG from both dark matter and modified gravity alternatives and provide a
clear set of tests for upcoming surveys.

\section*{References}

(References include SPARC database, lensing observations, cluster dynamics,
and prior CFG papers.)

\end{document}

