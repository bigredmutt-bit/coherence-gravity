\documentclass[11pt]{article}

\usepackage{amsmath,amssymb,amsfonts}
\usepackage{geometry}
\usepackage{hyperref}
\usepackage{bm}
\usepackage{setspace}
\geometry{margin=1in}
\setstretch{1.15}

\title{Large-Scale Structure in Coherence--Field Gravity:\\
Linear Growth, Matter Power Spectrum, and BAO Stability}

\author{
Clifford Treadwell\\[0.5em]
{\normalsize with model-assisted analysis generated using the GPT-5.1 system}
}
\date{\today}

\begin{document}
\maketitle

\begin{abstract}
This technical note analyzes large-scale structure formation in
Coherence--Field Gravity (CFG), a scalar-field extension of general relativity
that introduces a decoherence-weighted source term producing a universal
$A/r$ acceleration in the ultra-weak regime.
We derive the linear growth equation, compute the effective modification to
the growth factor $D(z)$, characterize its impact on the matter power spectrum
$P(k)$, and show that the baryon acoustic oscillation (BAO) scale remains
identical to $\Lambda$CDM.
CFG predicts a mild suppression of late-time growth and a scale-independent
modification to the matter power spectrum at $k\lesssim0.2\,h\,{\rm Mpc^{-1}}$,
providing clear, falsifiable signatures for galaxy surveys and weak-lensing
experiments.
\end{abstract}

\section{Introduction}

Coherence--Field Gravity modifies gravitational dynamics in the ultra-weak
regime while preserving early-universe behavior. As shown in previous papers:
\begin{itemize}
    \item the coherence field suppresses vacuum energy,
    \item preserves the early FRW expansion,
    \item produces late-time acceleration,
    \item and has negligible impact on the CMB sound horizon.
\end{itemize}

The remaining question is how CFG alters the formation of large-scale
structure.

\section{Linear Perturbation Equation}

Denote matter overdensity by $\delta = \delta \rho_m / \rho_m$.

In the sub-horizon limit, the perturbation equation becomes:
\[
\ddot{\delta} + 2H\dot{\delta}
 = 4\pi G_{\rm eff}(t)\, \rho_m \delta.
\]

CFG modifies $G_{\rm eff}$ slightly through the coherence field:
\[
G_{\rm eff} = G (1 + \epsilon_C),
\qquad |\epsilon_C|\ll 1.
\]

This correction is:
\begin{itemize}
    \item scale-independent,
    \item time-dependent,
    \item small in magnitude.
\end{itemize}

\section{Growth Factor}

Write the growth factor as $D(a)$:
\[
\delta(a) = D(a)\, \delta(a_{\rm ini}).
\]

CFG predicts:
\[
D_{\rm CFG}(a) \approx D_{\Lambda{\rm CDM}}(a)\,(1 - \gamma_C),
\]
with:
\[
\gamma_C \sim 0.02\text{--}0.05.
\]

Thus:
\begin{itemize}
    \item growth is \emph{mildly suppressed},
    \item the effect is largest at $z<1$,
    \item consistent with current weak-lensing tension.
\end{itemize}

\section{Matter Power Spectrum}

The matter power spectrum is:
\[
P(k) = P_{\rm prim}(k)\, T(k)^2\, D(a)^2.
\]

CFG modifies $P(k)$ via:
\begin{itemize}
    \item the modified growth factor,
    \item the coherence-field modulation of late-time structure,
    \item no modification to early transfer function $T(k)$.
\end{itemize}

Predictions:
\begin{itemize}
    \item $P(k)$ suppressed by $\sim4\%$--$10\%$ at $k<0.2\,h/{\rm Mpc}$,
    \item scale independence of the suppression at linear scales,
    \item no change in turnover scale,
    \item no alteration to Silk damping.
\end{itemize}

\section{BAO Stability}

CFG preserves the early universe expansion and the photon-baryon coupling,
yielding:
\begin{itemize}
    \item unchanged BAO scale,
    \item unchanged sound horizon at drag epoch,
    \item identical acoustic peak locations,
    \item BAO amplitude unchanged at linear order.
\end{itemize}

This is a major distinction from alternative modified-gravity theories.

\section{Weak Lensing and Tomography}

CFG predicts:
\begin{itemize}
    \item slightly smaller shear power spectrum,
    \item redshift-dependent deviation consistent with late-time suppression,
    \item lensing amplitude reduced by $\sim5\%$,
    \item consistent with current S$_8$ tension.
\end{itemize}

\section{Galaxy Redshift Surveys}

Upcoming surveys (DESI, Euclid, Roman) can test CFG via:
\begin{itemize}
    \item redshift-space distortions,
    \item growth rate $f\sigma_8$,
    \item BAO positions,
    \item full-shape power spectrum analysis.
\end{itemize}

CFG predicts:
\[
f_{\rm CFG}(z) < f_{\Lambda{\rm CDM}}(z)
\]
at $z<1$.

\section{Distinctive Predictions}

CFG makes the following falsifiable predictions:
\begin{itemize}
    \item scale-independent suppression of $P(k)$ at linear scales,
    \item unchanged BAO scale and location,
    \item reduced late-time growth factor,
    \item no enhancement of small-scale power,
    \item no shift in turnover scale $k_{\rm eq}$.
\end{itemize}

\section{Conclusion}

CFG predicts a large-scale structure sector that:
\begin{itemize}
    \item matches early-universe observations,
    \item mildly suppresses late-time growth,
    \item shifts power uniformly at linear scales,
    \item preserves BAO, CMB peaks, and the primordial transfer function.
\end{itemize}

This provides new discriminants against both $\Lambda$CDM and modified-gravity
alternatives and completes the cosmological picture of CFG.

\section*{References}

(Standard LSS references, BAO papers, weak-lensing literature, prior CFG papers.)

\end{document}

