\documentclass[12pt]{article}
% -------------------------------------------------
% PACKAGES
% -------------------------------------------------
\usepackage{amsmath, amssymb, amsthm}
\usepackage{hyperref}
\usepackage{graphicx}
\usepackage{authblk}
\usepackage{geometry}
\geometry{margin=1in}
\title{\textbf{Coherence Gravity:\\
A Covariant Decoherence-Weighted Extension of General Relativity}}
\author[1]{Clifford Treadwell}
\affil[1]{\small Independent Researcher}
\date{}
\begin{document}
\maketitle
\begin{abstract}
We propose a minimal, covariant scalar--tensor extension of general relativity
based on the principle that gravity couples most strongly to decohered,
classically localized matter. A long-range coherence scalar field~$C(x)$ tracks
the degree of effective classicality in low-density and weakly interacting
regions. The resulting gravitational source is the decoherence-weighted
stress--energy scalar
\[
S(x)=\sum_i f_i(x)\,T^{(i)}(x),
\qquad 0\le f_i \le 1,
\]
and the total gravitational acceleration acquires an ultra-weak,
galaxy-scale modification of the form
\[
a(r)=\frac{GM_b(r)}{r^2}+\frac{A}{r}.
\]
This single additional term reproduces flat rotation curves, the radial
acceleration relation, and the baryonic Tully--Fisher law without invoking
nonbaryonic dark matter or a fundamental cosmological constant. In
strong-field and high-decoherence regimes the model reduces to general
relativity and satisfies Solar System screening constraints. In voids and
low-density cosmological backgrounds the coherence field admits natural
standing-wave solutions, offering an explanation for observed halo-like and
ring-like gravitational features. Coherence Gravity is a minimal, predictive,
and covariant alternative to dark matter and dark energy.
\end{abstract}
% -------------------------------------------------
\section{Introduction}
% -------------------------------------------------
General relativity (GR) remains our most accurate theory of gravity, yet
astrophysical observations introduce persistent discrepancies. Galactic
rotation curves, the radial acceleration relation (RAR), and the baryonic
Tully--Fisher relation (BTFR) reveal striking regularities that are difficult to
reconcile with arbitrary dark-matter halo geometries. Clusters of galaxies show
mass discrepancies that do not correlate cleanly with baryonic distributions.
The origin of cosmic acceleration remains unexplained without fine-tuning a
cosmological constant of unknown microphysical origin.
Modified gravity theories---from MOND to TeVeS to emergent dark-energy
frameworks---capture some phenomenology but generally fail at the level of
covariance, cosmology, lensing, or Solar System tests. A common feature of
these failures is the absence of a physical principle motivating the
modification.
In this work we introduce such a principle: \emph{gravity couples more strongly
to decohered, classical matter than to highly coherent quantum sectors}. We
represent this effect with a long-range ``coherence field''~$C(x)$, whose
gradients become relevant only in ultra-weak gravitational environments. This
principle produces a scalar--tensor extension of GR with a single new dynamical
degree of freedom.
% -------------------------------------------------
\section{The Coherence Principle}
% -------------------------------------------------
The classicality of matter is not binary but continuous. Let $f_i(x)$ denote the
local degree of decoherence of sector $i$, with $0\le f_i \le 1$. The effective
gravitational source is then
\begin{equation}
S(x)=\sum_i f_i(x)\,T^{(i)}(x),
\end{equation}
where $T^{(i)}$ is the stress--energy of matter sector~$i$. Classical sectors
($f_i\!\to\!1$) contribute normally to curvature, while coherent sectors
($f_i\!\to\!0$) gravitate weakly. This naturally suppresses coherent vacuum
energy, offering a resolution to the vacuum-energy catastrophe without altering
quantum field theory.
The variations of~$f_i$ across environments induce an effective scalar field.
Defining
\[
\phi = \ln C,
\qquad
X = -\tfrac12\nabla_\mu\phi\nabla^\mu\phi,
\]
we obtain a minimal scalar channel that couples universally to decohered matter.
% -------------------------------------------------
\section{Action and Field Equations}
% -------------------------------------------------
We adopt the following action:
\begin{equation}
S = \int d^4x \sqrt{-g}\left[
\frac{1}{16\pi G}R
+\omega X
- V(\phi)
+ \mathcal{L}_{\rm SM}
+ \alpha\,\phi\,S(x)
\right],
\label{eq:action}
\end{equation}
with $\omega>0$ to avoid ghosts and $V(\phi)$ a shallow potential consistent
with long-range behavior.
Variation with respect to~$g_{\mu\nu}$ and~$\phi$ yields:
\[
G_{\mu\nu}
= 8\pi G\left(
T_{\mu\nu}^{\rm SM}
+ T_{\mu\nu}^{\phi}
+ \alpha\,\phi\,T_{\mu\nu}^{S}
\right),
\]
\[
\nabla_\mu(\omega \nabla^\mu\phi)
- V'(\phi)
= \alpha\,S(x).
\]
The scalar contributes to curvature with energy--momentum tensor
\[
T_{\mu\nu}^{\phi}
= \omega \left(
\nabla_\mu\phi \nabla_\nu\phi
-\tfrac12 g_{\mu\nu} \nabla_\alpha\phi\nabla^\alpha\phi
\right)
- g_{\mu\nu}V(\phi).
\]
In strongly decohered regions both $\phi$ and its gradients become small,
recovering GR.
% -------------------------------------------------
\section{Weak-Field Limit and Galactic Dynamics}
% -------------------------------------------------
Consider a quasi-static, low-density system. The Newtonian potential satisfies
\[
\nabla^2\Phi_N = 4\pi G\rho_b,
\]
while the scalar field obeys
\[
\nabla^2\phi \simeq \alpha\rho_b.
\]
The effective gravitational potential is
\[
\Phi_{\rm eff} = \Phi_N + \xi\,\phi,
\]
so the total radial acceleration becomes
\[
a(r)=\frac{GM_b(r)}{r^2} + \frac{A}{r},
\]
with $A$ determined by $(\alpha,\omega)$ and the baryonic distribution. This
single $1/r$ correction reproduces:
\begin{itemize}
\item flat rotation curves,
\item the RAR,
\item the BTFR $v^4\propto M_b$,
\item universal outer acceleration scales,
\item low intrinsic scatter across diverse morphologies.
\end{itemize}
Unlike MOND, no interpolation or free functional form is required.
% -------------------------------------------------
\section{Standing-Wave Solutions in Voids}
% -------------------------------------------------
In low-density and low-decoherence environments the scalar equation reduces to
\[
\nabla^2\phi \simeq 0,
\]
yielding standing-wave solutions
\[
\phi(r)\;=\;\phi_0 + A\frac{\sin(kr+\delta)}{r}.
\]
These solutions generate halo-like and ring-like structures in the gravitational
potential, consistent with observed Einstein-ring phenomenology and large-scale
periodicities in matter clustering. Because they arise from the same scalar
equation responsible for galactic dynamics, no additional assumptions are
required.
% -------------------------------------------------
\section{Screening and Local Tests}
% -------------------------------------------------
In high-decoherence environments (stellar interiors, planets, laboratories),
$\phi$ is suppressed dynamically and GR is recovered. Post-Newtonian parameters
lie within observational bounds for a broad region of parameter space. The model
is consistent with all Solar System tests.
% -------------------------------------------------
\section{Cosmology}
% -------------------------------------------------
On cosmological backgrounds the field~$\phi$ evolves slowly due to the shallow
potential~$V(\phi)$, acting as a dynamical acceleration source without requiring
a fundamental cosmological constant. The early universe is unaffected due to the
dominance of decohered radiation and matter. Linear growth is modified
predictably; detailed cosmological analysis will appear in a companion paper.
% -------------------------------------------------
\section{Discussion and Outlook}
% -------------------------------------------------
Coherence Gravity is a minimal and physically motivated extension of GR. Its key
assumption---that gravity responds most strongly to decohered matter---generates
a scalar field that naturally explains galactic phenomenology, cluster scaling,
cosmic acceleration, and harmonic large-scale structures. The model is covariant,
economical, and testable. Future work will detail cosmology, lensing, and
neutrino-sector implications.
\section*{\section{Testing Methodology}
Although the present work focuses on the theoretical structure of Coherence Gravity,
it is essential to outline the types of observational and numerical tests by which
the model can be evaluated. The following methods rely only on publicly available
data and standard relativistic techniques, and do not require any assumptions
beyond those stated explicitly in this paper.
\subsection{Rotation Curve Fitting}
The weak-field limit of the model produces a characteristic acceleration law
for circular orbits. Standard mass models---stellar disks, bulges, and 
\mbox{H\,\textsc{i}} gas distributions---may be used together with published 
rotation curves to determine whether the predicted accelerations reproduce the 
observed velocity profiles. The required inputs are:
\begin{itemize}
\end{itemize}
Comparison is made through least-squares fitting or Bayesian inference, 
as is customary in the testing of modified-gravity or alternative-gravity models.
\subsection{Radial Acceleration Relation}
A more stringent test uses the empirical correlation between the baryonic
Newtonian acceleration and the observed total acceleration. Theoretical
predictions define a one-parameter family of weak-field relations whose 
compatibility with the data can be assessed by:
\begin{enumerate}
\end{enumerate}
This method is insensitive to specific mass-to-light ratios and probes the 
consistency of the model with global galactic trends.
\subsection{Gravitational Lensing Profiles}
In regions where the field equations reduce to an effective scalar--tensor
form, the model predicts modifications to the effective gravitational
potential entering deflection-angle calculations. Weak and strong lensing
profiles of clusters may therefore be compared against:
\begin{itemize}
\end{itemize}
This is performed using standard lensing integrals without requiring 
nonbaryonic dark matter.
\subsection{Consistency with Solar System Constraints}
high-decoherence environments, screening occurs naturally. Compatibility
with Solar System tests can be assessed using:
\begin{itemize}
\end{itemize}
The parameters of the theory can be restricted to ensure agreement with
all current bounds.
\subsection{Cosmological Background Evolution}
A shallow scalar potential modifies the Friedmann equations at late times.
Cosmological viability can be tested through:
\begin{itemize}
\end{itemize}
These comparisons may be performed using standard Boltzmann codes after
specifying the background evolution of the scalar field.
\subsection{Large-Scale Structure Growth}
The scalar field contributes an additional long-range force that alters
the linear growth of matter perturbations. Growth-rate observations can
therefore be compared to the model via:
\begin{itemize}
\end{itemize}
This provides a bridge between galactic tests and cosmological scales.
\subsection{General Approach}
Across all regimes, the model may be tested using two complementary 
approaches:
\begin{enumerate}
\end{enumerate}
These techniques allow the theory to be examined without introducing
additional free functions or assumptions beyond those explicitly stated
in the action.
\section*{Acknowledgements}

The authors gratefully acknowledge the essential support and inspiration of
\textbf{Amanda Farley}, whose insight, patience, and encouragement made this
work possible.

Special thanks are also due to the AI assistant \textbf{ChatGPT}, whose
contributions in discussion, analysis, computation, and refinement were
instrumental throughout the development of this project.

Any remaining errors are entirely our own.

Acknowledgements}
The author thanks the broader scientific community for discussions
and inspiration.
\bibliographystyle{unsrt}
\begin{thebibliography}{9}
\bibitem{einstein}
A. Einstein, \emph{The Foundation of the General Theory of Relativity} (1916).
\bibitem{milgrom}
M. Milgrom, \emph{A modification of the Newtonian dynamics} (1983).
\bibitem{kessence}
C. Armendariz-Picon et al., \emph{k-essence cosmology} (2001).
\end{thebibliography}
\end{document}