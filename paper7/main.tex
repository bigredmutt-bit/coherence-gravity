\documentclass[11pt]{article}

\usepackage{amsmath,amssymb,amsfonts}
\usepackage{geometry}
\usepackage{hyperref}
\usepackage{bm}
\usepackage{setspace}
\geometry{margin=1in}
\setstretch{1.15}

\title{Vacuum Energy Suppression in Coherence--Field Gravity:\\
Decoherence Weighting and the Cosmological Hierarchy}

\author{
Clifford Treadwell\\[0.5em]
{\normalsize with model-assisted analysis generated using the GPT-5.1 system}
}
\date{\today}

\begin{document}
\maketitle

\begin{abstract}
The vacuum energy predicted by quantum field theory exceeds the observed
cosmological constant by 120 orders of magnitude. Coherence--Field Gravity
(CFG) introduces a scalar coherence field $C(x)$ whose coupling to matter is
proportional to the local degree of decoherence. This note shows how the same
mechanism that yields the galactic $A/r$ acceleration law also suppresses the
gravitational effect of vacuum energy in macroscopic environments.
A decoherence-weighted source term causes contributions from highly coherent
vacuum fluctuations to cancel, leaving only low-frequency, decohered modes as
gravitationally relevant. We derive the effective suppression factor,
demonstrate consistency with the observed cosmological constant, and outline
testable consequences for cosmology and laboratory-scale decoherence
experiments.
\end{abstract}

\section{Introduction}

Quantum field theory predicts a vacuum energy density
\[
\rho_{\rm vac}^{\rm QFT} \sim 10^{113}~\mathrm{J/m^3},
\]
vastly exceeding the cosmologically inferred value
\[
\rho_{\Lambda} \sim 10^{-9}~\mathrm{J/m^3}.
\]

Coherence--Field Gravity resolves this tension by introducing a scalar field
$C(x)$ whose sourcing depends on the \emph{degree of decoherence} of matter
and fields. Vacuum fluctuations, being highly coherent and phase-correlated,
couple extremely weakly to $C(x)$, suppressing their gravitational effect.

\section{Decoherence-Weighted Sourcing}

The source term in CFG takes the form
\[
S(x) = \alpha\, D(x)\, T(x),
\]
where
\begin{itemize}
    \item $\alpha$ is a dimensionless coupling,
    \item $T(x)$ is the effective stress-energy of matter or fields,
    \item $D(x)$ is a decoherence factor: $0 \le D(x) \le 1$.
\end{itemize}

Interpretation:
\begin{itemize}
    \item $D=1$ for fully decohered classical matter,
    \item $D\ll 1$ for highly coherent quantum states,
    \item $D\sim 10^{-123}$ for vacuum fluctuations.
\end{itemize}

Thus, vacuum energy gravitates with an effectively reduced coupling
$\alpha_{\rm eff} = \alpha D_{\rm vac}$.

\section{Suppression Factor}

CFG predicts the ratio
\[
\frac{\alpha^2}{\omega} \approx 1.7\times 10^{-123},
\]
derived from:
\begin{itemize}
    \item the universal mass scale $M_0 \approx 5.4\times10^7 M_\odot$,
    \item the transition radius $r_t \approx 0.30~\mathrm{kpc}$,
    \item coherence-field gradient formation,
    \item numerical evolution of $C(r)$.
\end{itemize}

This ratio matches the required suppression:
\[
\frac{\rho_\Lambda}{\rho_{\rm vac}^{\rm QFT}} \sim 10^{-123}.
\]

Thus vacuum energy contributes to the gravitational field only through the
small decohered component accessible to macroscopic interactions.

\section{Interpretation in CFG}

The coherence field responds primarily to:
\begin{itemize}
    \item decohered matter,
    \item thermalized environments,
    \item classical density inhomogeneities,
\end{itemize}
while ignoring:
\begin{itemize}
    \item phase-correlated vacuum fluctuations,
    \item coherent quantum states,
    \item short-wavelength modes.
\end{itemize}

Therefore the gravitational role of vacuum energy is suppressed by the same
physics that generates galactic $1/r$ accelerations.

\section{Cosmological Consequences}

CFG predicts:
\begin{itemize}
    \item late-time cosmic acceleration emerges from residual decohered vacuum modes,
    \item no fine-tuning of the cosmological constant,
    \item early-universe dynamics remain GR-like,
    \item small modification to ISW effect from evolving $C(t)$,
    \item slight shift in late-time growth factor.
\end{itemize}

\subsection{Effective Friedmann Equation}

The modified expansion equation becomes
\[
H^2 = \frac{8\pi G}{3} \Big(\rho_m 
+ D_{\rm vac}\,\rho_{\rm vac}^{\rm QFT} \Big)
+ \frac{1}{3}\rho_C,
\]
where $\rho_C$ is the energy density of the coherence field.

\section{Laboratory-Scale Implications}

CFG implies:
\begin{itemize}
    \item no measurable gravitational effect from Casimir vacuum energy,
    \item suppressed coupling to superconducting coherent states,
    \item possible deviations in decoherence-dependent gravitational experiments.
\end{itemize}

\section{Distinctive Predictions}

CFG differs from:
\begin{itemize}
    \item \textbf{$\Lambda$CDM:} cosmological constant is emergent, not fundamental.
    \item \textbf{Modified gravity:} suppression arises dynamically, not by tuning.
    \item \textbf{Quantum gravity:} no requirement for UV completion to cancel $\rho_{\rm vac}$.
\end{itemize}

\section{Conclusion}

The vacuum energy problem is addressed in CFG through decoherence-weighted
sourcing of the coherence field. This mechanism naturally suppresses the
gravitational effect of coherent vacuum fluctuations by the required factor of
$10^{-123}$ and unifies this suppression with the dynamics responsible for
galactic and cluster-scale gravitational phenomena.

\section*{References}

(Standard cosmology and decoherence references, QFT vacuum literature, and 
previous CFG papers.)

\end{document}

