\section{Deep-Regime Behavior and the Emergence of a Logarithmic Potential}
\label{sec:deep}

The nonlinear scalar flux relation~\eqref{eq:flux} determines the structure of
the far-field solution for the coherence scalar $C(r)$ and thereby the
asymptotic form of the effective gravitational acceleration. In this section we
analyze the exterior, deep-scalar regime in detail, demonstrating that the
coherence response generates a logarithmic potential and a corresponding $1/r$
acceleration with amplitude $\sqrt{GMa_0}$.

\subsection{Asymptotic Structure of the Scalar Gradient}

Outside the baryonic mass distribution, the scalar flux relation takes the form
\begin{equation}
r^2\left(1+\kappa\,C'^2\right)C' = K(M),
\qquad
K(M) = -f'(C_{\rm bg})\,M,
\label{eq:flux_repeat}
\end{equation}
with $\kappa \equiv \lambda_4/\Lambda^4$ and $C' \equiv dC/dr$. Two distinct
regimes arise depending on the relative strength of the canonical and
quartic-gradient terms.

\paragraph*{(1) Canonical regime: $|\kappa C'^2|\ll 1$.}
The quartic term is negligible, and Eq.~\eqref{eq:flux_repeat} reduces to
\begin{equation}
C'(r) \simeq \frac{K(M)}{r^2},
\end{equation}
implying $C\propto r^{-1}$ and reproducing the Newtonian $1/r^2$ structure of
the scalar gradient. This regime dominates at small radii or for low enclosed
baryonic mass.

\paragraph*{(2) Nonlinear regime: $|\kappa C'^2|\gg 1$.}
When the quartic-gradient term dominates, the scalar equation reduces to
\begin{equation}
\kappa\,C'^3 \simeq \frac{K(M)}{r^2},
\end{equation}
yielding the power-law solution
\begin{equation}
C'(r) \propto \left(\frac{M}{r^2}\right)^{1/3}
\propto M^{1/3} r^{-2/3}.
\end{equation}
The transition between regimes occurs where $\kappa C'^2\sim 1$, which gives
\begin{equation}
r_{\rm trans}(M) \propto \sqrt{M},
\end{equation}
in agreement with both the B2/D2 numerical results and the empirical MOND
scaling. This mass scaling arises directly from the conserved scalar flux and
does not require dark matter or modifications to the Poisson equation.

\subsection{Effective Gravitational Potential in the Deep Regime}

The effective gravitational potential felt by baryonic matter follows from the
modified Poisson equation~\eqref{eq:PoissonWeak}. In the deep-scalar regime the
coherence energy density $\rho_C$ and the interaction term $\rho_{\rm int}$
dominate, yielding
\begin{equation}
\nabla^2\Phi
\simeq 4\pi G\left(
\tfrac{1}{2}|\nabla C|^2
+ \tfrac{\lambda_4}{4\Lambda^4}|\nabla C|^4
+ \rho_{\rm int}
\right),
\label{eq:PoissonDeep}
\end{equation}
whose radial dependence is inherited from $C'(r)$.

For $|\nabla C|\propto r^{-2/3}$, the dominant terms in
Eq.~\eqref{eq:PoissonDeep} scale as $r^{-4/3}$ and $r^{-8/3}$, which integrate
to produce an asymptotically logarithmic gravitational potential,
\begin{equation}
\Phi(r) \simeq \Phi_0 + A\ln(r/r_0),
\end{equation}
with
\begin{equation}
A = \sqrt{GMa_0},
\label{eq:A_repeat}
\end{equation}
where $a_0$ is the emergent acceleration scale determined in
Sec.~\ref{sec:SPARC}. Differentiating gives the asymptotic acceleration
\begin{equation}
a(r) = -\Phi'(r) = \frac{A}{r},
\end{equation}
consistent with both the B2/D2 simulations and the empirical baryonic
Tully--Fisher relation.

\subsection{Summary of Deep-Regime Behavior}

The deep-scalar regime exhibits three key features:
\begin{enumerate}
\item \textbf{Conserved scalar flux:}\;
$r^2(1+\kappa C'^2)C'=K(M)$, with $K(M)\propto M$.
\item \textbf{Mass-scaling of the transition radius:}\;
$r_{\rm trans}\propto\sqrt{M}$, matching the MOND empirical law.
\item \textbf{Logarithmic potential and $1/r$ tail:}\;
the quartic-gradient term generates a potential
$\Phi\simeq A\ln r$ with amplitude $A=\sqrt{GMa_0}$.
\end{enumerate}
These features arise solely from the dynamics encoded in the Lagrangian
\eqref{eq:fullL}. The consistency between analytic structure, numerical B2/D2
solutions, and galactic data forms a central pillar of the coherence-field
framework.

