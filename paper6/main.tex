\documentclass[11pt]{article}

\usepackage{amsmath,amssymb,amsfonts}
\usepackage{geometry}
\usepackage{hyperref}
\usepackage{bm}
\usepackage{setspace}
\geometry{margin=1in}
\setstretch{1.15}

\title{Falsifiable Predictions of Coherence--Field Gravity:\\
A Clear Experimental and Observational Test Suite}

\author{
Clifford Treadwell\\[0.5em]
{\normalsize with model-assisted analysis generated using the GPT-5.1 system}
}
\date{\today}

\begin{document}
\maketitle

\begin{abstract}
A scientific theory must make predictions that can be tested and potentially
proven wrong. Coherence--Field Gravity (CFG) proposes a universal $A/r$
acceleration term sourced by baryonic decoherence, producing flat rotation
curves and cluster-scale gravitational effects without particle dark matter.
This technical note enumerates the concrete observational and experimental
tests capable of falsifying CFG. These tests span Solar System dynamics,
galactic structure, gravitational lensing, wide binaries, cluster dynamics,
and cosmological probes. Each prediction is derived from the core CFG
acceleration law and is independent of galaxy-by-galaxy tuning or free
parameters.
\end{abstract}

\section{Introduction}

CFG predicts the following acceleration law:
\[
a(r) = a_N(r) + \frac{A}{r},
\qquad
a_N(r)=\frac{GM(r)}{r^2}.
\]

The strength of CFG is its simplicity: the $1/r$ term is universal and arises
from the dynamical evolution of the coherence field $C(r)$. This simplicity
produces strong, falsifiable observational signatures.

This note catalogs the most decisive tests.

\section{Solar System Tests}

In the Solar System:
\[
a_N \gg \frac{A}{r}.
\]

CFG predicts:
\begin{itemize}
    \item no measurable deviations from Newtonian gravity at $\lesssim 100$ AU,
    \item perihelion precession of Mercury unchanged within current uncertainty,
    \item Pioneer anomaly not explained by CFG (providing a falsification point),
    \item no anomalous acceleration of outer planets.
\end{itemize}

\subsection{Falsification Criterion}

\emph{Any detected deviation from Newtonian/GR dynamics within Solar System
precision---that matches a $1/r$ form in the range 1--100~AU---falsifies CFG.}

\section{Wide Binary Stars (Gaia DR4+)}

CFG predicts a mild enhancement over Newtonian velocities at separations
\[
r \approx 5{,}000\text{--}20{,}000~\mathrm{AU}.
\]

Distinctive signatures:
\begin{itemize}
    \item smooth transition in acceleration (no sharp MOND-like threshold),
    \item velocity dispersion scaling as $\sigma \propto r^{-1/2}$,
    \item no break at $10^{-11}\,\mathrm{m/s^2}$.
\end{itemize}

\subsection{Falsification Criterion}

\emph{If Gaia DR4+ finds a MOND-like sharp acceleration threshold, or purely
Newtonian scaling at $r>10{,}000$ AU, CFG is falsified.}

\section{Galactic Rotation Curves}

CFG predicts:
\begin{itemize}
    \item universal transition radius $r_t \approx 0.30$ kpc,
    \item $a(r)$ approaching $A/r$ at large radius,
    \item reduced scatter relative to MOND or NFW models,
    \item no galaxy-by-galaxy parameter tuning.
\end{itemize}

\subsection{Falsification Criterion}

\emph{If a statistically significant subset of galaxies exhibits a transition
radius outside the narrow range $0.25$--$0.35$ kpc, CFG fails.}

\section{Gravitational Lensing}

CFG predicts a lensing potential:
\[
\Phi(r)= -\frac{GM(r)}{r} + A \ln r.
\]

Consequences:
\begin{itemize}
    \item stronger-than-baryonic shear,
    \item shallower-than-NFW outer shear profile,
    \item no central cusps in the convergence map.
\end{itemize}

\subsection{Falsification Criterion}

\emph{Detection of NFW-like $1/r^3$ outer shear in systems where baryonic mass
is well constrained falsifies CFG.}

\section{Galaxy Clusters}

CFG predicts:
\begin{itemize}
    \item velocity dispersions consistent with inferred cluster mass,
    \item absence of massive dark halos,
    \item mildly enhanced lensing arcs,
    \item shallower mass profiles than NFW.
\end{itemize}

\subsection{Falsification Criterion}

\emph{Accurate reconstruction of an NFW-like halo profile in a relaxed cluster
without invoking dark matter falsifies CFG.}

\section{CMB and Cosmology}

CFG predicts:
\begin{itemize}
    \item negligible modification to early-universe expansion,
    \item altered late-time growth factor,
    \item slightly modified ISW effect,
    \item no shift in acoustic peak locations.
\end{itemize}

\subsection{Falsification Criterion}

\emph{A measured growth factor inconsistent with CFG's modified Friedmann
equation falsifies the model.}

\section{What CFG Cannot Mimic}

CFG does \emph{not} predict:
\begin{itemize}
    \item NFW-like cusps,
    \item subhalo structure,
    \item sharp MOND transitions,
    \item deviations in Solar System accelerations,
    \item galaxy-by-galaxy parameter variation.
\end{itemize}

Observation of any of these, in clean data, would challenge the framework.

\section{Summary of Falsification Points}

CFG is ruled out if:
\begin{itemize}
    \item Solar System shows $1/r$ deviations,
    \item wide binaries follow MOND-style acceleration thresholds,
    \item transition radii vary widely across SPARC galaxies,
    \item clusters consistently require NFW halos,
    \item CMB growth factor contradicts CFG dynamics.
\end{itemize}

\section{Conclusion}

CFG provides a clean, falsifiable set of predictions across gravitational
observables. These tests are accessible with existing and upcoming datasets.
The framework stands or falls on these measurable signatures.

\section*{References}

(Standard astrophysical references, SPARC database, Gaia papers, etc.)

\end{document}

