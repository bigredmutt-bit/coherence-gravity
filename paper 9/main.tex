\documentclass[11pt]{article}

\usepackage{amsmath,amssymb,amsfonts}
\usepackage{geometry}
\usepackage{hyperref}
\usepackage{bm}
\usepackage{setspace}
\geometry{margin=1in}
\setstretch{1.15}

\title{Gravitational Waves in Coherence--Field Gravity:\\
Propagation, Dispersion, and Observational Signatures}

\author{
Clifford Treadwell\\[0.5em]
{\normalsize with model-assisted analysis generated using the GPT-5.1 system}
}
\date{\today}

\begin{document}
\maketitle

\begin{abstract}
Coherence--Field Gravity (CFG) modifies the gravitational potential through a
scalar coherence field $C(x)$ whose gradient produces an additional $A/r$
acceleration in the weak-field regime. This paper analyzes the impact of these
modifications on gravitational wave propagation. Because CFG preserves the
Einstein--Hilbert action at leading order and the coherence field couples
primarily to decohered matter, gravitational waves propagate through vacuum
with near-GR behavior. We derive the effective wave equation, show that
dispersion is negligible for LIGO/Virgo frequencies, and outline potential
signatures at cosmological distances due to coherence-field backreaction.
CFG predicts GR-consistent waveforms with small late-time deviations in
amplitude and phase accumulation over gigaparsec baselines.
\end{abstract}

\section{Introduction}

CFG introduces a scalar coherence field $C(x)$ that modifies gravitational
dynamics in the ultra-weak regime. The question addressed here is whether such
a modification affects gravitational wave (GW) propagation.

The key insights:
\begin{itemize}
    \item high-coherence vacuum states couple weakly to $C(x)$,
    \item gravitational waves propagate through vacuum,
    \item therefore the coherence field does not significantly perturb GW dynamics.
\end{itemize}

\section{Perturbative Framework}

Write the metric as:
\[
g_{\mu\nu} = \eta_{\mu\nu} + h_{\mu\nu},
\]
with $|h_{\mu\nu}|\ll1$.

The coherence field is decomposed as:
\[
C(x) = C_0 + \delta C(x),
\]
where $C_0$ is the cosmological background.

\section{Linearized Equations}

In CFG, the GW equation becomes:
\[
\square h_{\mu\nu}
= 16\pi G\, T_{\mu\nu}^{\rm GW}
+ S_{\mu\nu}[C],
\]
with $S_{\mu\nu}$ encoding coherence-field backreaction.

Because:
\begin{itemize}
    \item gravitational waves are coherent quantum excitations of spacetime,
    \item and $C(x)$ couples weakly to coherent states,
\end{itemize}
we obtain:
\[
S_{\mu\nu}[C] \approx 0.
\]

Thus:
\[
\square h_{\mu\nu} \simeq 0.
\]

\section{Propagation Speed}

CFG predicts:
\[
c_{\rm GW} = c,
\]
consistent with:
\begin{itemize}
    \item GW170817 timing constraints,
    \item multimessenger bounds.
\end{itemize}

\section{Dispersion}

To leading order:
\[
\omega^2 = k^2.
\]

Small corrections arise from cosmological drift in $C(t)$:
\[
\omega^2 = k^2 \left( 1 + \epsilon_C \right),
\qquad
|\epsilon_C| \ll 10^{-15}.
\]

Thus dispersion is:
\begin{itemize}
    \item unmeasurable for LIGO/Virgo,
    \item potentially measurable for LISA at $10^{-4}$--$10^{-1}$ Hz,
    \item strongest at Gpc propagation distances.
\end{itemize}

\section{Amplitude and Phase Modifications}

The coherence field modifies the background expansion:
\[
H(t) = H_{\Lambda{\rm CDM}} + \Delta H_C.
\]

Consequences for gravitational waves:
\begin{itemize}
    \item amplitude decay differs slightly due to modified luminosity distance,
    \item phase accumulation differs at the $\sim10^{-4}$ level over Gpc scales,
    \item possible signatures in standard-siren cosmography.
\end{itemize}

\subsection{Luminosity Distance}

CFG predicts:
\[
d_L^{\rm CFG}(z) = d_L^{\Lambda{\rm CDM}}(z)\, (1 + \Delta_C),
\]
with
\[
|\Delta_C| \sim 10^{-3}.
\]

\section{Strong-Field Effects}

The coherence field does not strongly couple in:
\begin{itemize}
    \item black hole mergers,
    \item neutron star mergers,
    \item ringdown behavior.
\end{itemize}

Thus:
\[
\text{Waveform templates from GR remain valid.}
\]

\section{Distinctive Predictions}

CFG predicts:
\begin{itemize}
    \item no change to GW propagation speed,
    \item no birefringence,
    \item tiny dispersion at cosmological distances,
    \item modified luminosity distances for standard sirens,
    \item phase drift detectable by LISA with long baselines.
\end{itemize}

These are clean and falsifiable.

\section{Conclusion}

CFG yields gravitational wave propagation nearly identical to GR in all
presently accessible regimes. Deviations appear only over cosmological
distances due to background coherence-field evolution, providing a set of
testable signatures for next-generation GW observatories.

\section*{References}

(GR perturbation theory texts, LIGO/Virgo papers, LISA forecasts, prior CFG papers.)

\end{document}

