\documentclass[11pt]{article}

\usepackage{amsmath,amssymb,amsfonts}
\usepackage{geometry}
\usepackage{hyperref}
\usepackage{bm}
\usepackage{setspace}
\geometry{margin=1in}
\setstretch{1.15}

\title{The Coherence--Field Gravity Research Program:\\
Overview, Structure, and Roadmap}

\author{
Clifford Treadwell\\[0.5em]
{\normalsize with model-assisted analysis generated using the GPT-5.1 system}
}
\date{\today}

\begin{document}
\maketitle

\begin{abstract}
This document provides an overview of the complete Coherence--Field Gravity
(CFG) research suite, summarizing the ten papers that establish the theoretical,
numerical, observational, and cosmological foundations of the framework.
CFG introduces a scalar coherence field $C(x)$ whose decoherence-weighted
coupling to matter produces a universal $A/r$ acceleration in the ultra-weak
regime and suppresses vacuum energy by the required factor of $10^{-123}$.
This overview outlines the structure of the research program, describes the
role of each individual paper, and presents a roadmap for future development
in theory, observation, and simulation.
\end{abstract}

\section{Introduction}

The Coherence--Field Gravity program was developed to unify galactic
dynamics, cluster phenomenology, vacuum energy suppression, and cosmological
acceleration within a single scalar-field extension of general relativity.
The framework introduces:
\begin{itemize}
    \item a coherence field $C(x)$,
    \item decoherence-weighted sourcing of gravity,
    \item a universal acceleration term $A/r$,
    \item and a natural suppression of vacuum energy.
\end{itemize}

This document summarizes the ten foundational papers comprising the initial
release of CFG.

\section{Summary of the CFG Papers}

\subsection*{Paper 1: Foundations and Equations}
Defines the Lagrangian, derives the field equations, and establishes the
coherence-field contribution to gravity. Introduces the $A/r$ acceleration
and the universal mass scale $M_0$.

\subsection*{Paper 2: Unified Galactic and Cosmological Interpretation}
Develops the phenomenological consequences across galaxies, clusters, and
cosmology. Derives the transition radius $r_t$ and explores large-scale
solutions.

\subsection*{Paper 3: Methodology of Human--AI Co-Discovery}
Documents the collaborative process behind CFG, outlining the reasoning loop,
failure modes, and stabilization mechanisms that enabled rapid development.

\subsection*{Paper 4: Numerical Evolution and Stability}
Details the solver architecture and high-resolution simulations that confirm
the emergence of the $1/r$ gradient in baryonic environments.

\subsection*{Paper 5: Observational Predictions}
Enumerates the consequences for galaxies, clusters, lensing, wide binaries,
and dwarf galaxies, identifying testable signatures of CFG.

\subsection*{Paper 6: Falsifiable Tests}
Presents the decisive observational and experimental criteria capable of
falsifying CFG, establishing the framework as empirically grounded.

\subsection*{Paper 7: Vacuum Energy Suppression Mechanism}
Shows how decoherence-weighted coupling suppresses vacuum energy by
$10^{-123}$, resolving the cosmological hierarchy.

\subsection*{Paper 8: CFG Cosmology}
Derives the FRW equations, demonstrates unshifted BAO scales, and accounts
for late-time acceleration without a fundamental $\Lambda$.

\subsection*{Paper 9: Gravitational Waves}
Analyzes GW propagation, concluding that GR behavior is preserved except for
small cosmological-scale effects detectable by LISA.

\subsection*{Paper 10: Large-Scale Structure}
Derives the linear growth equation and matter power spectrum. Predicts mild
late-time suppression consistent with weak-lensing tension.

\section{Program Structure}

CFG is organized around three core pillars:
\begin{itemize}
    \item \textbf{Field Theory:} Papers 1, 2, 7.
    \item \textbf{Numerical and Observational:} Papers 4, 5, 6, 9, 10.
    \item \textbf{Cosmology:} Papers 2, 7, 8, 10.
\end{itemize}

Paper 3 provides a methodological foundation for future hybrid
human--AI research programs.

\section{Roadmap for Future Work}

Future directions include:
\begin{itemize}
    \item 2D and 3D numerical simulations,
    \item lensing simulations using the CFG potential,
    \item cluster-scale pressure profile modeling,
    \item cosmological parameter fitting,
    \item exploration of alternative potentials $V(C)$,
    \item laboratory decoherence-gravity experiments.
\end{itemize}

\section{Conclusion}

This overview consolidates the ten-paper CFG research suite into a unified
program. The structure provides a coherent basis for future development and
external scientific engagement.

\section*{References}
(References to the ten CFG papers, GR texts, cosmology, and decoherence
literature.)

\end{document}

