\begin{abstract}
We develop a covariant scalar-field extension of gravity in which a single
coherence field $C(x)$ responds nonlinearly to the trace of the baryonic
stress--energy. The theory is defined by a minimal Lagrangian containing a
canonical kinetic term, a quartic-gradient interaction, and a shallow potential.
In the weak-field, static limit the scalar equation admits a conserved nonlinear
flux relation,
$r^2(1+\kappa C'^2)C' = K(M)$ with $K(M)\propto M_{\rm bar}$. This relation
yields the observed baryonic Tully--Fisher scaling, the mass--acceleration
correlation, and an asymptotic $1/r$ acceleration with amplitude
$A=\sqrt{GMa_0}$, where $a_0\simeq10^{-12}\,\mathrm{m\,s^{-2}}$ emerges
dynamically. Using the SPARC database, we show that the model reproduces
rotation curves across 175 disk galaxies with a single parameter set and
residuals comparable to or better than MOND and $\Lambda$CDM halo fits.

On cluster scales, the same quartic-gradient dynamics generate an extended
scalar-induced ``effective halo'' with mass $M_C\sim(3$--$7)\,M_{\rm bar}$,
reproducing hydrostatic X-ray masses, SZ pressure profiles, and weak-lensing
maps without collisionless dark matter. The lensing morphology of merging
clusters, including the Bullet Cluster, follows naturally from the coherent
matter coupling.

Cosmologically, the coherence field behaves as a radiation-like component at
early times, suppressing vacuum energy and preserving CMB acoustic peaks, while
its late-time slow roll drives cosmic acceleration without requiring a
cosmological constant. The result is a unified framework in which galactic
dynamics, cluster phenomenology, gravitational lensing, and cosmic acceleration
arise from the nonlinear response of a single scalar field.
\end{abstract}

