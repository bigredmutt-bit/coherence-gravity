\section{Structure Formation and CMB Predictions}
\label{sec:StructureFormation}

Having established the homogeneous FRW behavior of the coherence field, we now
examine linear perturbations around the cosmological background. Our goal is to
determine whether the coherence-field model supports the observed growth of
large-scale structure and produces CMB anisotropies consistent with current
measurements. We show that the quartic-gradient term plays a decisive role: it
suppresses scalar clustering at early times, preserving standard radiation
acoustics, while the canonical regime at late times supports gravitational
instability and structure growth on galactic and cluster scales.

\subsection{Perturbations of the Coherence Field}

We expand the scalar field as
\begin{equation}
C(t,\mathbf{x})=\bar{C}(t)+\delta C(t,\mathbf{x}),
\end{equation}
and adopt Newtonian gauge for the metric perturbations,
\begin{equation}
ds^2=-(1+2\Psi)\,dt^2 + a(t)^2(1-2\Phi)\,d\mathbf{x}^2.
\end{equation}
Linearizing Eq.~\eqref{eq:scalarEOM} yields
\begin{align}
&\left(1 - 3\kappa\dot{\bar{C}}^{\,2}\right)\ddot{\delta C}
+ 3H\left(1 - \kappa\dot{\bar{C}}^{\,2}\right)\dot{\delta C}
- \frac{1}{a^2}\bigl(1 + \kappa\dot{\bar{C}}^{\,2}\bigr)\nabla^2\delta C
\notag\\[4pt]
&\qquad\qquad
+ V''(\bar{C})\,\delta C
= f''(\bar{C})\,\bar{T}\,\delta C
+ f'(\bar{C})\,\delta T,
\label{eq:deltaC}
\end{align}
where the coefficient of the Laplacian defines an effective sound speed,
\begin{equation}
c_s^2
= \frac{1 + \kappa\dot{\bar{C}}^{\,2}}{1 - 3\kappa\dot{\bar{C}}^{\,2}}.
\label{eq:cs2}
\end{equation}

\paragraph*{Early Universe: quartic-gradient regime.}
When $|\dot{\bar{C}}|$ is large, the quartic term dominates:
\begin{equation}
c_s^2 \simeq \frac{1}{3}.
\end{equation}
The coherence field behaves like a relativistic component with $w_C\simeq 1/3$
and does not cluster on sub-horizon scales. Thus the C-field leaves the standard
radiation-driven acoustic oscillations of the CMB essentially unmodified.

\paragraph*{Late Universe: canonical regime.}
When $\kappa\dot{\bar{C}}^{\,2}\ll 1$, the sound speed approaches
\begin{equation}
c_s^2 \simeq 1,
\end{equation}
as in quintessence models. Perturbations propagate at the speed of light. Since
the coherence field couples only to the trace of the matter stress--energy, its
direct influence on large-scale matter perturbations is small.

\subsection{Growth of Matter Perturbations}

The matter overdensity $\delta_m$ satisfies the modified growth equation
\begin{equation}
\ddot{\delta}_m
+ 2H\dot{\delta}_m
- 4\pi G_{\rm eff}(k,t)\,\bar{\rho}_m\,\delta_m
= 0,
\label{eq:growth_eq}
\end{equation}
where
\begin{equation}
G_{\rm eff}(k,t)
= G\bigl(1 + \Delta_C(k,t)\bigr)
\end{equation}
collects the scalar contribution via the perturbed coherence energy and
interaction terms. Using Eq.~\eqref{eq:deltaC}, two characteristic regimes
emerge:

\begin{itemize}
\item \textbf{Radiation era (quartic domination):}  
$\delta C$ oscillates with a relativistic sound speed, suppressing scalar
clustering. Thus $\Delta_C\simeq 0$, and the matter growth rate matches standard
cosmology.

\item \textbf{Matter era (canonical regime):}  
the scalar perturbation can enhance the Newtonian potential on small scales,
\begin{equation}
\Delta_C(k,t) \simeq
\begin{cases}
0, & k \ll k_{\rm coh},\\[4pt]
\mathcal{O}(1), & k \gtrsim k_{\rm coh},
\end{cases}
\end{equation}
where $k_{\rm coh}$ corresponds to the transition radius
$r_{\rm trans}(M)\propto\sqrt{M}$ that characterizes galactic halos. Modes that
enter the scalar-dominated regime experience enhanced gravitational instability,
consistent with the observed early formation of galactic disks.
\end{itemize}

\subsection{Implications for the CMB}

The coherence field influences the CMB in three ways:

\paragraph*{(1) Acoustic peak structure.}
Since $c_s^2\simeq 1/3$ and $w_C\simeq 1/3$ in the radiation era, the coherence
field behaves like an additional radiation-like component with suppressed energy
density. It leaves the standard acoustic oscillations essentially unchanged,
preserving the observed CMB peak structure.

\paragraph*{(2) Integrated Sachs--Wolfe (ISW) effect.}
During late-time slow roll ($w_C\simeq -1$), the evolving coherence potential
produces an ISW effect similar to dynamical dark energy. The predicted amplitude
lies within current observational uncertainty.

\paragraph*{(3) Absence of early ISW enhancement.}
Unlike TeVeS and other MOND-inspired relativistic theories, the coherence field
does not produce large early gravitational potentials that would strongly
enhance the early ISW signal. The radiation-like scaling of $\rho_C$ at high
redshift prevents the ``early ISW bump'' that challenges most MOND extensions.

\subsection{Structure Formation Summary}

The time-dependent behavior of the coherence field yields a consistent picture
of cosmic structure formation:
\begin{enumerate}
\item \textbf{Radiation epoch:} quartic-gradient domination forces the
coherence field to behave like radiation, suppressing early clustering and
preserving the standard CMB peak structure.

\item \textbf{Matter epoch:} canonical-gradient and potential terms dominate,
allowing the coherence field to enhance gravitational potentials on galactic
scales while leaving large-scale modes nearly unchanged.

\item \textbf{Late-time epoch:} slow roll of $C$ produces cosmic acceleration
with $w_C\simeq -1$, generating an ISW signature consistent with current CMB
data.

\item \textbf{Overall:} the scalar yields MOND-like galactic dynamics and nearly
standard early-universe cosmology—a combination difficult to achieve in MOND,
TeVeS, or $\Lambda$CDM without parameter tuning.
\end{enumerate}
Thus the coherence-field Lagrangian reproduces the key features of the early
Universe, supports the formation of galaxies and clusters at the observed
epochs, and remains consistent with CMB observations across all angular scales.

