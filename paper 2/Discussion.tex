\section{Unified Discussion and Synthesis}
\label{sec:discussion}

The coherence-field framework developed in this work provides a single,
covariant, and dynamically consistent explanation for gravitational phenomena
across an enormous range of physical scales—from kiloparsec galactic disks to
megaparsec galaxy clusters to cosmological horizons.  In this section we draw
together the results of the preceding analysis and articulate the unifying
principles that make this possible.

\subsection{A Single Scalar Degree of Freedom Across All Scales}

The model is defined by a single scalar field $C(x)$ with a canonical kinetic
term, a quartic-gradient interaction, and a soft potential $V(C)$.  No new
particles, vector fields, or hidden sectors are introduced.  All phenomenology
arises from:
\begin{enumerate}
\item the nonlinear scalar flux relation,
\item the environment-sensitive coupling to the trace of the matter
stress--energy, and
\item the background evolution of $C$ across cosmic time.
\end{enumerate}
This minimality is a major departure from alternative gravity theories such as
TeVeS, scalar–vector–tensor models, or high-derivative frameworks, which
require multiple new fields to fit individual regimes (galaxies, clusters, or
cosmology) and often struggle to make them mutually consistent.

\subsection{Galactic Dynamics Without Dark Matter}

The nonlinear flux relation,
\begin{equation}
r^2\left(1+\kappa C'^2\right)C' = K(M),
\end{equation}
determines the entire galactic phenomenology.  
For baryonic mass $M$, the transition to the deep-scalar regime occurs at a
radius
\begin{equation}
r_{\rm trans}\propto\sqrt{M},
\end{equation}
and the asymptotic gravitational acceleration acquires a logarithmic tail,
\begin{equation}
a(r) = \frac{\sqrt{GMa_0}}{r}.
\end{equation}
These two relations correspond exactly to the observed Tully–Fisher scaling and
the baryonic mass–acceleration relation revealed through SPARC.  
Crucially, $a_0$ is not inserted by hand: it emerges from the same parameters
$(\Lambda_4,f)$ that govern the scalar dynamics in the Lagrangian.

\subsection{Clusters and the Emergence of an Effective Halo}

On cluster scales, the coherence field continues to respond to the baryonic
mass, but the large gas fraction and extended mass distribution dramatically
enhance the scalar contribution.  
The quartic-gradient term remains important well beyond galactic radii, giving
rise to an extended ``effective halo'' with mass
\begin{equation}
M_C(r)\sim(3\text{--}7)M_{\rm bar}(r).
\end{equation}
This single dynamical contribution simultaneously explains:
\begin{itemize}
\item the hydrostatic equilibrium of the intracluster medium,
\item the observed X-ray temperatures and pressure profiles,
\item the consistency of the baryon fraction at large radii, and
\item the lensing maps of merging clusters, including the Bullet Cluster.
\end{itemize}
These phenomena remain severe challenges for MOND and relativistic MOND
extensions unless additional matter is introduced ad hoc.

\subsection{Cosmology Without a Cosmological Constant}

The coherence field evolves naturally from quartic-gradient domination at early
times to a slow-rolling regime at late times.  
In the early Universe ($|\dot{C}|\gg 1$) the field behaves as a radiation-like
component with $w_C\simeq 1/3$, suppressing vacuum energy and leaving the CMB
peak structure unchanged.  
In the late Universe the potential $V(C)$ dominates and $w_C\rightarrow -1$,
driving cosmic acceleration without requiring a cosmological constant.

The same parameters $(\Lambda_4,f)$ that reproduce the SPARC scaling also
determine the onset and magnitude of cosmic acceleration.  
No fine-tuning or new mass scales are introduced beyond those already
constrained by galactic dynamics.

\subsection{Absence of Free Functions or Tuned Interpolation}

A key virtue of the coherence-field model is the \emph{absence of any free
interpolation function}.  
In MOND this function must be chosen empirically and often varies between
authors.  
In TeVeS, multiple free functions are required to stabilize the scalar and
vector sectors.

Here:
\begin{itemize}
\item the flux equation alone determines the mass scaling,
\item the quartic gradient alone determines the deep-scalar behavior,
\item the background coupling alone determines $a_0$, and
\item the rolling potential alone determines late-time acceleration.
\end{itemize}
The interpolation behavior between Newtonian and deep-scalar regimes emerges
directly from the dynamics of the scalar field.

\subsection{A Unified Interpretation}

Taken together, the results of this work provide a unified interpretation of
gravity in which dark-matter-like phenomena arise from the nonlinear dynamics
of a single scalar field sourced by decohered baryonic matter:
\begin{itemize}
\item \textbf{Galaxies} exhibit MOND-like dynamics and tight mass–acceleration
relations.
\item \textbf{Clusters} exhibit extended, smooth lensing halos consistent with
observations.
\item \textbf{Cosmology} exhibits early-time radiation-like behavior and
late-time acceleration, all from the same field.
\end{itemize}

In this picture dark matter is not a particle species but rather a macroscopic,
collective response of the coherence field to the distribution and environment
of baryonic matter.  
The same scalar field that shapes galaxy rotation curves also drives cosmic
acceleration, and does so without requiring the introduction of any new
fundamental scales or arbitrary functional degrees of freedom.

\subsection{Outlook}

The coherence-field model opens several avenues for further research:
\begin{enumerate}
\item Nonlinear structure formation simulations with the full scalar dynamics.
\item CMB multipole predictions including the late ISW effect.
\item Detailed modeling of strong-lensing cluster mergers.
\item Constraints from gravitational waves and multimessenger events.
\end{enumerate}
These directions will sharpen the observational signatures of the coherence
field and may provide decisive tests that distinguish it from both $\Lambda$CDM
and conventional modified-gravity theories.

The results presented here show that a single, covariant scalar field can unify
galactic, cluster, and cosmological gravitational phenomena in a manner that is
both theoretically minimal and observationally robust.
