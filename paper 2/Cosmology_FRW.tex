\section{Cosmology and FRW Dynamics}
\label{sec:FRW}

The coherence-field Lagrangian~\eqref{eq:fullL} provides a covariant framework
that extends naturally to cosmological scales and admits a homogeneous FRW
background. In this section we derive the associated FRW equations, identify
the effective energy density and pressure of the coherence field, and show that
the quartic-gradient dynamics suppress early-time vacuum energy while producing
late-time cosmic acceleration without introducing a cosmological constant.

\subsection{Homogeneous Coherence Field in FRW}

We adopt a spatially flat Friedmann--Robertson--Walker metric
\begin{equation}
ds^2 = -dt^2 + a(t)^2\,d\mathbf{x}^2,
\label{eq:FRWmetric}
\end{equation}
and take the background coherence field to depend only on time,
$C=C(t)$. The kinetic term reduces to
\begin{equation}
X = (\nabla C)^2 = -\dot{C}^{\,2}.
\end{equation}
The scalar Lagrangian becomes
\begin{equation}
\mathcal{L}_C
= \frac{1}{2}\dot{C}^{\,2}
-\frac{\lambda_4}{4\Lambda^4}\dot{C}^{\,4}
- V(C),
\end{equation}
where the quartic term now acts as a ``stiff'' correction to the canonical
kinetic energy.

\subsection{Effective Energy Density and Pressure}

For a $k$-essence Lagrangian $\mathcal{L}(C,X)$ the energy--momentum tensor is
\begin{equation}
T^{(C)}_{\mu\nu}
= \mathcal{L}_X \nabla_\mu C \nabla_\nu C - g_{\mu\nu}\mathcal{L},
\end{equation}
with $\mathcal{L}_X\equiv\partial\mathcal{L}/\partial X$. Specializing to FRW
gives
\begin{align}
\rho_C &= \frac{1}{2}\dot{C}^{\,2}
+\frac{3\lambda_4}{4\Lambda^4}\dot{C}^{\,4}
+ V(C),
\label{eq:rhoC_FRW}
\\[4pt]
p_C &= \frac{1}{2}\dot{C}^{\,2}
+\frac{\lambda_4}{4\Lambda^4}\dot{C}^{\,4}
- V(C).
\label{eq:pC_FRW}
\end{align}
The quartic term contributes positively to the energy density while raising the
pressure more mildly, a key property for suppressing vacuum energy at early
times.

\subsection{Modified Friedmann Equations}

Including the coherence field, the Friedmann equations read
\begin{align}
H^2 &= \frac{1}{3M_{\rm Pl}^2}
\left(\rho_m + \rho_r + \rho_C\right),
\label{eq:Friedmann1}
\\[4pt]
\frac{\ddot{a}}{a}
&= -\frac{1}{6M_{\rm Pl}^2}
\left(\rho_m + \rho_r + \rho_C + 3p_C\right).
\label{eq:Friedmann2}
\end{align}
The scalar equation of motion, derived from Eq.~\eqref{eq:scalarEOM}, becomes
\begin{equation}
\left(1 - 3\kappa\dot{C}^{\,2}\right)\ddot{C}
+ 3H\left(1 - \kappa\dot{C}^{\,2}\right)\dot{C}
+ V'(C)
= f'(C)\,T_{\rm bg},
\label{eq:CEOM_FRW}
\end{equation}
where $T_{\rm bg} = -\rho_m + 3p_m$. During radiation domination
$T_{\rm bg}\simeq 0$, while during matter domination $T_{\rm bg}\simeq -\rho_m$.

\subsection{Early-Time Behavior and Vacuum-Energy Suppression}

At early times, when $|\dot{C}|$ is large, the quartic term dominates:
\begin{equation}
\rho_C \simeq \frac{3\lambda_4}{4\Lambda^4}\dot{C}^{\,4},
\qquad
p_C \simeq \frac{\lambda_4}{4\Lambda^4}\dot{C}^{\,4}.
\end{equation}
The coherence field therefore obeys
\begin{equation}
w_C \equiv \frac{p_C}{\rho_C}
\simeq \frac{1}{3},
\end{equation}
and redshifts like radiation, $\rho_C\propto a^{-4}$, regardless of the bare
potential $V(C)$. This automatically suppresses vacuum energy at high redshift,
preserving standard big bang cosmology and avoiding fine-tuning of $V(C)$ in
the early Universe.

\subsection{Late-Time Behavior and Accelerated Expansion}

As the Universe expands, $\dot{C}$ redshifts and the canonical kinetic term and
potential dominate over the quartic piece. The scalar equation reduces to
\begin{equation}
\ddot{C} + 3H\dot{C} + V'(C)
\simeq f'(C)\,\rho_m,
\end{equation}
and the field enters a slow-roll regime. The energy density is then
\begin{equation}
\rho_C \simeq V(C) + \frac{1}{2}\dot{C}^{\,2},
\end{equation}
with an equation of state $w_C\simeq -1$ provided $V(C)$ is sufficiently flat.
Crucially, late-time acceleration arises \emph{without} introducing a
cosmological constant: the vacuum energy is dynamically generated by the
scalar potential and the matter-coupling term.

\subsection{Cosmological Interpretation}

The cosmological coherence field exhibits a natural two-regime structure:
\begin{enumerate}
\item \textbf{Early Universe:}  
quartic-gradient dominance drives $\rho_C\propto a^{-4}$, suppressing bare
vacuum energy and preserving standard early cosmology.
\item \textbf{Late Universe:}  
canonical and potential terms dominate, yielding a slow-rolling scalar with
$w_C\simeq -1$ and generating cosmic acceleration without a cosmological
constant.
\end{enumerate}

The same quartic-gradient term that produces MOND-like galactic dynamics also
ensures early-time radiation-like behavior, while the slow-roll potential that
drives late-time acceleration is consistent with the Lagrangian parameters
$(\Lambda_4,f)$ determined from galactic data alone. Thus the coherence-field
model provides a unified relativistic framework for galactic, cluster, and
cosmological phenomena without particle dark matter or a tuned cosmological
constant.

