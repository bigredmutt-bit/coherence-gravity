\section{Relation to MOND Interpolation Functions}
\label{sec:mond_mu}

Modified Newtonian Dynamics (MOND) is commonly expressed in terms of an
interpolation function $\mu(x)$ that relates the total acceleration $a$ to the
Newtonian prediction $a_{\rm N}$ according to
\begin{equation}
a\,\mu\!\left(\frac{a}{a_0}\right) = a_{\rm N},
\label{eq:mond_mu_def}
\end{equation}
or, equivalently, in terms of a $\nu$-function defined by
\begin{equation}
a = \nu\!\left(\frac{a_{\rm N}}{a_0}\right)a_{\rm N},
\qquad
y \equiv \frac{a_{\rm N}}{a_0}.
\label{eq:mond_nu_def}
\end{equation}
The empirical success of MOND at galactic scales is encoded in the limiting
behavior
\begin{equation}
\mu(x)\to 1 \quad (x\gg 1),
\qquad
\mu(x)\to x \quad (x\ll 1),
\label{eq:mond_limits}
\end{equation}
or, equivalently,
\begin{equation}
\nu(y)\to 1 \quad (y\gg 1),
\qquad
\nu(y)\to y^{-1/2} \quad (y\ll 1).
\end{equation}
In this section we show that the coherence-field model reproduces these limits
and closely tracks standard MOND interpolation functions across the full SPARC
acceleration range.

\subsection{Effective Interpolation Function from the Coherence Field}

In the coherence-field framework the total acceleration is
\begin{equation}
a_{\rm tot}(r)
= a_{\rm N}(r) + a_{\rm coh}(r),
\qquad
a_{\rm N}(r) = \frac{GM_{\rm bar}(r)}{r^2},
\end{equation}
with the deep-regime coherence acceleration given by
\begin{equation}
a_{\rm coh}(r)
= \frac{\sqrt{G M a_0}}{r},
\end{equation}
as derived in the preceding sections. For a given baryonic mass profile, this
defines an \emph{effective} MOND $\nu$-function:
\begin{equation}
a_{\rm tot}
= \nu_{\rm eff}(y)\,a_{\rm N},
\qquad
y \equiv \frac{a_{\rm N}}{a_0}.
\label{eq:nu_eff_def}
\end{equation}

In the regime where the scalar tail contributes significantly but the Newtonian
term is not negligible, the coherence-field solution is well approximated by
\begin{equation}
a_{\rm tot}(r)
\simeq a_{\rm N}(r)
+ \sqrt{a_{\rm N}(r)a_0},
\label{eq:a_tot_approx}
\end{equation}
implying
\begin{equation}
\nu_{\rm eff}(y)
= 1 + y^{-1/2}.
\label{eq:nu_eff_approx}
\end{equation}
This form is not imposed; it emerges from the nonlinear scalar flux relation
and the B2/D2 dynamics.

\subsection{Limiting Behavior and Comparison with Standard MOND Forms}

The effective coherence-field interpolation function satisfies the MOND limits:
\begin{align}
y \gg 1:\quad
&\nu_{\rm eff}(y) = 1 + y^{-1/2} \simeq 1,
\qquad
a_{\rm tot}\simeq a_{\rm N},
\\[4pt]
y \ll 1:\quad
&\nu_{\rm eff}(y) \sim y^{-1/2},
\qquad
a_{\rm tot} \simeq \sqrt{a_0\,a_{\rm N}},
\end{align}
reproducing the deep-MOND scaling.  
The corresponding effective $\mu$-function,
\begin{equation}
\mu_{\rm eff}(x)
= \frac{a_{\rm N}}{a_{\rm tot}}
= \frac{1}{\nu_{\rm eff}(y)},
\qquad
x=\frac{a_{\rm tot}}{a_0},
\end{equation}
satisfies the standard limits~\eqref{eq:mond_limits}:
\begin{equation}
\mu_{\rm eff}(x)\to 1 \quad (x\gg 1),
\qquad
\mu_{\rm eff}(x)\to x \quad (x\ll 1).
\end{equation}

Common MOND choices include
\begin{equation}
\mu_{\rm simple}(x) = \frac{x}{1+x},
\qquad
\mu_{\rm standard}(x) = \frac{x}{\sqrt{1+x^2}}.
\end{equation}
Across the SPARC acceleration range,
$\nu_{\rm eff}(y)=1+y^{-1/2}$ is numerically very close to these forms,
consistent with the small log-RMS scatter ($\simeq 0.08$ dex) in
Fig.~\ref{fig:nu_vs_aN}.  
The coherence-field model therefore matches both the asymptotic limits and the
empirical shape of the MOND interpolation favored by SPARC.

\subsection{Conceptual Difference from MOND}

Despite the close phenomenology, the underlying mechanisms differ
fundamentally:
\begin{enumerate}
\item In MOND, the $\mu(x)$ function is an \emph{input}—a phenomenological
interpolating rule built into a modified Poisson equation.
\item In the coherence-field theory, the effective interpolation emerges
\emph{dynamically} from a covariant Lagrangian with a single scalar degree of
freedom and a quartic gradient term. The interpolation function is a derived
quantity reflecting the scalar's nonlinear response to baryonic sources.
\end{enumerate}
This conceptual distinction is crucial: the coherence-field model provides a
field-theoretic origin for MOND-like behavior, embedding galactic phenomenology
within a unified relativistic and cosmological framework.

