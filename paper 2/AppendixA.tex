\appendix
\section{Mapping Between Numerical Solver Units and Physical Parameters}
\label{app:BDmapping}

In the quasi-static, weak-field limit the coherence field obeys the nonlinear
radial equation
\begin{equation}
\frac{1}{r^2}\frac{d}{dr}
\left[
r^2\left(1+\kappa C'^2\right)C'
\right]
= -f_1\,\rho(r),
\qquad
\kappa \equiv \frac{\lambda_4}{\Lambda^4},
\label{eq:scalar_radial_app}
\end{equation}
where $f_1 \equiv f'(C_{\rm bg})$ is the effective coherence--matter coupling
and primes denote radial derivatives. Integrating over the baryonic source
yields, for all radii exterior to the mass distribution,
\begin{equation}
r^2\left(1+\kappa C'^2\right)C'
= K_{\rm phys}(M),
\qquad
K_{\rm phys}(M) = -f_1\,M,
\label{eq:K_phys_app}
\end{equation}
demonstrating that the scalar flux constant is proportional to the enclosed
baryonic mass.

\subsection*{Nondimensionalization and the B2/D2 Code Variables}

The B2/D2 solver evolves a dimensionless field $\phi(x)$ using the scalings
\begin{equation}
C = C_0\,\phi,
\qquad
r = r_0\,x.
\end{equation}
Substituting into Eq.~\eqref{eq:K_phys_app} gives
\begin{equation}
r_0 C_0\,x^2\left[1+\Lambda_4\phi'^2(x)\right]\phi'(x)
= K_{\rm phys}(M),
\qquad
\Lambda_4 \equiv \kappa\,\frac{C_0^2}{r_0^2}.
\end{equation}
This motivates the definition of the \emph{dimensionless flux constant}
\begin{equation}
K_{\rm code}(M_0)
\equiv x^2\Bigl[1+\Lambda_4\phi'^2(x)\Bigr]\phi'(x),
\label{eq:Kcode_app}
\end{equation}
which becomes spatially constant outside the numerical density profile.

Thus the physical and numerical flux constants satisfy
\begin{equation}
K_{\rm phys}(M)
= r_0 C_0\,K_{\rm code}(M_0),
\qquad
M = M_0\,M_{\rm unit}.
\label{eq:phys_vs_code_app}
\end{equation}

The B2/D2 solver outputs a dimensionless parameter $f$ and accepts an input
mass $M_0$. Empirically, across all solver runs,
\begin{equation}
K_{\rm code}(M_0) = f\,M_0,
\label{eq:code_flux_scaling_app}
\end{equation}
and this combination also determines the numerical transition radius,
\begin{equation}
r_{\rm trans}(M_0) = f\,M_0.
\end{equation}
Combining Eqs.~\eqref{eq:K_phys_app}--\eqref{eq:code_flux_scaling_app}, the
physical coupling $f_1$ relates to the numerical parameter $f$ via
\begin{equation}
f_1
= -\,\frac{r_0 C_0}{M_{\rm unit}}\,f,
\end{equation}
so that the physical flux constant becomes
\begin{equation}
K_{\rm phys}(M)
= r_0 C_0\,f\,M_0
= -f_1\,M.
\end{equation}

\subsection*{Connection to the MOND Asymptotic Acceleration}

In the regime where the scalar sector dominates the gravitational response,
the observable acceleration matches the MOND deep-regime form
\begin{equation}
a(r) = \frac{\sqrt{GMa_0}}{r}.
\label{eq:mond_accel_app}
\end{equation}
Defining the physical amplitude
\begin{equation}
A_{\rm phys}(M)
\equiv \sqrt{GMa_0},
\end{equation}
and noting that the numerical code uses
\begin{equation}
r_0 = \sqrt{\frac{GM_{\rm unit}}{a_0}},
\qquad
a_{\rm code} = \frac{a_{\rm phys}}{a_0},
\end{equation}
the corresponding dimensionless amplitude is
\begin{equation}
A_{\rm code}(M_0)
= \frac{A_{\rm phys}(M)}{a_0\,r_0}
= \sqrt{M_0}.
\label{eq:A_code_app}
\end{equation}

Thus the B2/D2 solver precisely reproduces the MOND predictions that
\begin{itemize}
\item the amplitude of the logarithmic $1/r$ tail scales as $\sqrt{M}$, and
\item the transition radius satisfies $r_{\rm trans}\propto\sqrt{M}$.
\end{itemize}
Both properties follow directly from the nonlinear scalar flux relation
\eqref{eq:K_phys_app} combined with the quartic kinetic term.

