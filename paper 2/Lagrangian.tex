\section{The Coherence--Field Lagrangian}
\label{sec:lagrangian}

The dynamics of the metric $g_{\mu\nu}$ and the coherence scalar field $C(x)$
are determined by the covariant action
\begin{equation}
S = \int d^4x\,\sqrt{-g}\left[
\frac{M_{\rm Pl}^2}{2}R
-\frac{1}{2}(\nabla C)^2
-\frac{\lambda_4}{4\Lambda^4}\!\left[(\nabla C)^2\right]^2
- V(C)
- f(C)\,T
+ \mathcal{L}_{\rm m}(\psi_i,g_{\mu\nu})
\right].
\label{eq:fullL}
\end{equation}
The scalar $C$ represents the coherence response of decohered (effectively
classical) matter and fields, and it couples to the trace of the matter
stress--energy tensor $T$ through the function $f(C)$. The quartic-gradient
term, parametrized by $\lambda_4/\Lambda^4$, is the unique second-order scalar
operator that survives weak-field consistency tests and produces the required
nonlinear scaling on galactic and cluster scales.

\subsection{Field Equations}

Variation with respect to $C$ yields the scalar equation of motion
\begin{equation}
\nabla_\mu\!\left[
\Bigl(1 + \frac{\lambda_4}{\Lambda^4}(\nabla C)^2\Bigr)\nabla^\mu C
\right]
- V'(C)
- f'(C)\,T
= 0,
\label{eq:scalarEOM}
\end{equation}
where $(\nabla C)^2 = g^{\mu\nu}\nabla_\mu C \nabla_\nu C$. The quartic term
introduces a nonlinear dependence on $(\nabla C)^2$, analogous to $X^2$
operators in k-essence theories but with a specific sign and coefficient chosen
to reproduce the observed galactic phenomenology.

Varying the action with respect to $g^{\mu\nu}$ produces the Einstein equations
\begin{equation}
M_{\rm Pl}^2\,G_{\mu\nu}
= T^{(m)}_{\mu\nu}
+ T^{(C)}_{\mu\nu}
+ T^{(\mathrm{int})}_{\mu\nu},
\label{eq:EinsteinFull}
\end{equation}
where $T^{(m)}_{\mu\nu}$ is the matter stress--energy tensor,
\begin{equation}
T^{(C)}_{\mu\nu}
=
\Bigl(-\tfrac{1}{2}-\tfrac{\lambda_4}{2\Lambda^4} X\Bigr)
\nabla_\mu C\,\nabla_\nu C
+ g_{\mu\nu}\left[
\tfrac{1}{2}X
+ \tfrac{\lambda_4}{4\Lambda^4}X^2
+ V(C)
\right],
\qquad 
X \equiv (\nabla C)^2,
\end{equation}
and $T^{(\mathrm{int})}_{\mu\nu}$ arises from the variation of the interaction
term $-f(C)T$. The latter acts as an effective renormalization of the matter
sector through the coherence coupling.

\subsection{Weak-Field, Quasi-Static, Nonrelativistic Limit}

For galactic systems the spacetime metric may be written as
$ds^2 = -(1+2\Phi)\,dt^2 + (1-2\Phi)\,d\mathbf{x}^2$, with nonrelativistic
matter ($T\simeq -\rho$) and negligible time derivatives of $C$. Under these
conditions the scalar equation \eqref{eq:scalarEOM} becomes
\begin{equation}
\nabla\cdot\!\left[
\Bigl(1+\kappa\,|\nabla C|^2\Bigr)\nabla C
\right]
= f'(C_{\rm bg})\,\rho,
\qquad
\kappa\equiv\frac{\lambda_4}{\Lambda^4},
\label{eq:scalarWeak}
\end{equation}
while the Einstein equation reduces to a modified Poisson equation
\begin{equation}
\nabla^2\Phi =
4\pi G\left(
\rho
+ \tfrac{1}{2}|\nabla C|^2
+ \tfrac{\lambda_4}{4\Lambda^4}|\nabla C|^4
+ V(C)
+ \rho_{\rm int}
\right).
\label{eq:PoissonWeak}
\end{equation}
Equations~\eqref{eq:scalarWeak} and \eqref{eq:PoissonWeak} form a closed,
second-order elliptic system governing the coupled metric and scalar field.

\subsection{Exterior Solution and Nonlinear Flux Conservation}

Outside the baryonic mass distribution ($\rho=0$), the scalar equation
\eqref{eq:scalarWeak} integrates exactly to
\begin{equation}
r^2\left(1+\kappa\,C'^2\right)C' = K(M),
\qquad
K(M) = -f'(C_{\rm bg})\,M,
\label{eq:flux}
\end{equation}
where $M$ is the enclosed baryonic mass. Equation~\eqref{eq:flux} shows that
$M$ acts as a conserved scalar flux. The relation implies a transition between
a canonical regime where $C'\propto r^{-2}$ and a nonlinear regime where
$C'\propto r^{-2/3}$ when $\kappa C'^2\simeq 1$. The associated transition
radius satisfies $r_{\rm trans}\propto \sqrt{M}$, reproducing the MOND
mass–radius scaling.

\subsection{Asymptotic Acceleration and the MOND Coefficient}

In the deep-scalar regime the effective gravitational acceleration acquires an
asymptotic logarithmic tail
\begin{equation}
a(r) = \frac{A}{r},
\qquad
A = \sqrt{GMa_0},
\label{eq:logTail}
\end{equation}
with $a_0$ determined numerically from galactic fits. This reproduces the
characteristic MOND prediction for the large-radius behavior of disk galaxies,
while both the transition radius and amplitude arise directly from the scalar
flux law \eqref{eq:flux}.

Thus the Lagrangian \eqref{eq:fullL}, together with its field equations and
weak-field reduction, constitutes a minimal covariant framework in which the
nonlinear scalar response naturally recovers the empirical MOND scaling
relations and provides a unified interpretation of galactic dynamics without
invoking particulate dark matter.

