\documentclass[11pt]{article}

\usepackage{amsmath,amssymb,amsfonts}
\usepackage{geometry}
\usepackage{hyperref}
\usepackage{bm}
\usepackage{setspace}
\geometry{margin=1in}
\setstretch{1.15}

\title{Cosmology of Coherence--Field Gravity:\\
FRW Dynamics, Late-Time Acceleration, and Structure Growth}

\author{
Clifford Treadwell\\[0.5em]
{\normalsize with model-assisted analysis generated using the GPT-5.1 system}
}
\date{\today}

\begin{document}
\maketitle

\begin{abstract}
This paper develops the cosmological sector of Coherence--Field Gravity (CFG),
a scalar-field extension of general relativity in which the coherence field
$C(x)$ couples to matter through decoherence-weighted sourcing.
The same field that produces the galactic $A/r$ acceleration term also
suppresses the gravitational effect of vacuum energy.
At cosmological scales, the coherence field contributes an effective energy
density and modifies the Friedmann equation while preserving early-universe
behavior.
We derive the FRW evolution equations, analyze late-time acceleration, obtain
the linear growth factor for structure formation, and confirm that acoustic
peak locations remain unshifted. The result is a unified cosmological picture
consistent with both large-scale structure and late-time acceleration without a
fundamental cosmological constant.
\end{abstract}

\section{Introduction}

Coherence--Field Gravity introduces a scalar field $C(x)$ whose coupling to
matter is controlled by a decoherence factor $D(x)$. As shown in previous
papers:
\begin{itemize}
    \item the same physics yields a universal $A/r$ galactic acceleration,
    \item suppresses vacuum energy by $\sim10^{-123}$,
    \item and modifies gravitational dynamics in the ultra-weak regime.
\end{itemize}

Here we extend CFG to cosmology.

\section{FRW Framework}

Assume a spatially flat FRW metric:
\[
ds^2 = -dt^2 + a(t)^2 d\vec{x}^{\,2}.
\]

The coherence field is homogeneous:
\[
C = C(t).
\]

Its energy density and pressure are:
\[
\rho_C = \frac{1}{2}\dot{C}^2 + V(C),
\qquad
p_C = \frac{1}{2}\dot{C}^2 - V(C).
\]

\section{Decoherence-Weighted Vacuum Energy}

Vacuum fluctuations couple as:
\[
\rho_{\rm vac}^{\rm eff} = D_{\rm vac}\, \rho_{\rm vac}^{\rm QFT},
\]
where $D_{\rm vac}\sim10^{-123}$.

Thus the effective cosmological constant becomes:
\[
\rho_\Lambda^{\rm eff} = D_{\rm vac}\,\rho_{\rm vac}^{\rm QFT}.
\]

\section{Modified Friedmann Equation}

The expansion rate is:
\[
H^2 = \frac{8\pi G}{3}
\left[ \rho_m 
+ \rho_r 
+ D_{\rm vac}\,\rho_{\rm vac}^{\rm QFT}
+ \rho_C \right].
\]

Because $D_{\rm vac}$ is extremely small,
$\rho_C$ drives late-time acceleration.

\section{Evolution of the Coherence Field}

$C(t)$ satisfies:
\[
\ddot{C} + 3H\dot{C} + V'(C) = S_{\rm cos},
\]
where the cosmological source $S_{\rm cos}$ reflects large-scale
decoherence from matter inhomogeneities.

Late-time behavior:
\begin{itemize}
    \item $\dot{C}$ becomes small,
    \item $V(C)$ dominates,
    \item producing accelerated expansion.
\end{itemize}

\section{Early-Universe Behavior}

When $\rho_m$ and $\rho_r$ dominate:
\begin{itemize}
    \item $S_{\rm cos}$ is negligible,
    \item $\rho_C$ is subdominant,
    \item $C(t)$ tracks minimally,
\end{itemize}
ensuring:
\begin{itemize}
    \item unshifted CMB acoustic peaks,
    \item standard nucleosynthesis,
    \item unaltered radiation-to-matter equality.
\end{itemize}

\section{Late-Time Acceleration}

At $z<1$:
\begin{itemize}
    \item $C(t)$ becomes slowly rolling,
    \item $\rho_C$ mimics a cosmological constant,
    \item but without fine-tuning.
\end{itemize}

The equation of state approaches
\[
w_C \approx -1,
\]
with small departures that may be observable.

\section{Linear Structure Growth}

Matter perturbations obey:
\[
\ddot{\delta} + 2H\dot{\delta}
 = 4\pi G_{\rm eff} \rho_m \delta,
\]
where $G_{\rm eff}$ differs slightly from $G$ due to coherence-field feedback.

CFG predicts:
\begin{itemize}
    \item reduced growth relative to $\Lambda$CDM,
    \item scale-independent modification,
    \item mild late-time suppression.
\end{itemize}

This can be tested with:
\begin{itemize}
    \item redshift-space distortions,
    \item weak-lensing tomography,
    \item Lyman-$\alpha$ forest.
\end{itemize}

\section{BAO Stability}

Because CFG preserves the early universe:
\begin{itemize}
    \item BAO scale is unchanged,
    \item sound horizon at drag epoch matches $\Lambda$CDM,
    \item no shift in acoustic peak locations.
\end{itemize}

\section{Predictions and Tests}

CFG predicts:
\begin{itemize}
    \item slightly smaller late-time growth factor,
    \item ISW effect modified at low multipoles,
    \item cosmic acceleration without $\Lambda$,
    \item no early-time deviations.
\end{itemize}

\section{Conclusion}

CFG provides a cosmological picture in which:
\begin{itemize}
    \item early-universe evolution matches GR,
    \item vacuum energy is dynamically suppressed,
    \item late-time acceleration arises naturally,
    \item linear growth is mildly reduced.
\end{itemize}

This unifies galactic, cluster, and cosmological phenomenology within a single
scalar-field framework.

\section*{References}

(Standard cosmology references, decoherence literature, prior CFG papers.)

\end{document}

